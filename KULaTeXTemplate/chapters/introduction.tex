\chapter{Introduction}\label{chapter:introduction}
\section{Figures and Referencing}
This section shows how to make figures and cross-reference them in the text.
\Cref{fig1} shows an important figure.
It is not a most to use \textbf{\textbackslash Cref\{labelName\}}; you still can use referencing using \textbf{Figure \textbackslash ref\{labelName\}}.
Both will have the same effect.
Figure \ref{fig2} uses the second referencing technique. 

\begin{figure}
	\centering
	\includegraphics[]{example-image-a} 
	\caption{Letter 'A' Sample Figure. $\sum_{i=0}^{1000} \prod_{j=1}^{300} i*j$ }\label{fig1}
\end{figure}

\begin{figure}
	\centering
	\includegraphics[scale=1.00]{example-image-b} 	
	\caption{Letter 'B' Sample Figure. $\mathcal{B}=\{b_1,b_2,...,b_n\}, |B|\ge1$. }\label{fig2}
\end{figure}

\lipsum[1]


\section{Tables and Referencing}
Tables are similar to figures. You reference them the same ways. 
\Cref{table1} shows a simple table.
Try to avoid using vertical line. 
Usually horizontal lines only  used.
You can also use \texttt{tabularx} to create your table, which give you a better way in controlling the width of your table. 
Example of using \texttt{tabularx} is shown in \Cref{table2}

\begin{table}
	\centering
	\caption{Table caption example using \texttt{tabular}}\label{table1}
\begin{tabular}{lll}
	\toprule
header 1 & header 2 & header 3 \\
\midrule
f1 & f2& f3\\
f1 & f2& f3\\
f1 & f2& f3\\
f1 & f2& f3\\
f1 & f2& f3\\
f1 & f2& f3\\
\bottomrule
\end{tabular}

\end{table}

\begin{table}[t!]
	\centering
		\caption{Table caption example using \texttt{tabularx}}\label{table2}
	\begin{tabularx}{\textwidth}{llX}
		\toprule
		header 1 & header 2 & header 3 \\
		\midrule
		f1 & f2& \lipsum[1]\\
		f1 & f2& Sample Long Text. \\
		f1 & f2& \textarabic{تجريب بالعربي}\\
		f1 & f2&Sample Long Text.\\
		f1 & f2& Sample Long Text.\\
		f1 & f2& Sample Long Text.\\
		\bottomrule
	\end{tabularx}

\end{table}

\lipsum[1]
